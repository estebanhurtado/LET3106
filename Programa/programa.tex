\documentclass[letterpaper,11pt]{article}
\usepackage[T1]{fontenc}
\usepackage[utf8]{inputenc}
\usepackage{lmodern}
\usepackage[spanish]{babel}
\usepackage[margin=2cm]{geometry}

\title{{\Large Facultad de Letras - Pontificia Universidad Católica de Chile } \\ Curso LET3106 - M\'etodos y t\'ecnicas de investigaci\'on}
\author{Profesor: Esteban Hurtado}

\newcommand{\chapBegin}{\begin{minipage}[t]{0.2\columnwidth}}
\newcommand{\chapEnd}{\end{minipage}}

\newcommand{\contBegin}{\begin{minipage}[t]{0.65\columnwidth}\begin{itemize}\setlength{\itemsep}{0pt}\setlength{\parskip}{0pt}}
\newcommand{\contEnd}{\end{itemize}\end{minipage} \\[0.25in] \\}

\begin{document}

\maketitle
%\tableofcontents

%\begin{abstract}
%\end{abstract}

\section{Descripci\'on del curso}

El curso es una introducci\'on a la metodolog\'ia de la investigaci\'on cient\'ifica en ling\"u\'istica. Se abordan t\'ecnicas cuantitativas y cualitativas con \'enfasis en las primeras. Se menciona una variedad de m\'etodos com�nmente conocidos. De ellos, s\'olo algunos se abordan en profundidad por ser representativos y por su amplia aplicabilidad. 

No se asumen conocimientos previos sobre estad\'istica. Los aspectos num\'ericos se tratan desde lo m\'as b\'asico y sin mayor profundidad de la estrictamente necesaria. Con todo esto se persigue dotar al alumno de un conjunto elemental de herramientas que le permita enfrentar con comodidad un amplio rango de preguntas cient\'ificas, evaluar investigaciones cr\'iticamente y enfrentar el estudio personal de otros t\'opicos que resulten relevantes para sus investigaciones.

\section{Objetivos}

\begin{itemize}
  \item Dotar al alumno de herramientas b\'asicas que le permitan enfrentar satisfactoriamente proyectos de investigaci\'on metodol\'ogicamente sencillos.
    
  \item Proporcionar una mirada ``desde dentro'' del proceso de investigaci\'on, con el fin de comprender y evaluar investigaciones cr\'iticamente.

  \item Aportar con elementos que enriquezcan las opiniones que cada alumno se forme sobre controversias relacionadas con la investigaci\'on cient\'ifica.
  
  \item Preparar para el estudio personal de otros t\'opicos similares.

\end{itemize}


\section{Contenidos}

\begin{enumerate}
  \item \emph{Panorama de la metodolog\'ia de la investigaci\'on cient\'ifica}
  
  Presentaci\'on general de los temas a abordar. Discusi\'on acerca de la naturaleza de las preguntas cient\'ificas y el rol de las herramientas metodol\'ogicas a la hora de buscarles respuestas. 
  
  \item \emph{Estad\'istica descriptiva aplicada a la investigaci\'on cuantitativa}.\label{it:desc}
  
Conceptos y t\'ecnicas fundamentales para el uso de la cuantificaci\'on en investigaci\'on.
  
  \begin{itemize}
    \item Estad\'isticos descriptivos esenciales y su rol en la metodolog\'ia cuantitativa.
    \item Comunicaci\'on de estad\'isticos descriptivos.
  \end{itemize}

  \item \emph{Introducci\'on a la estad\'istica inferencial}.\label{it:inf}
  
La estad\'istica inferencial es de gran ayuda a la hora de juzgar si los patrones que se observan en una muestra son representativos de lo que ocurre en una poblaci\'on general. Se presentan las t\'ecnicas b\'asicas y se muestras c\'omo estas ayudan a responder preguntas cient\'ificas y a comunicar resultados.

  \begin{itemize}
    \item Estrategia de an\'alisis en estad\'istica inferencial.
    \item Contraste de dos muestras con la prueba $t$ de Student.
    \item An\'alisis de datos de diferentes dise\~nos experimentales mediante an\'alisis de la varianza.
    \item Prueba $\chi^2$
    \item Comunicaci\'on de resultados
    \item Visualizaci\'on de datos.
  \end{itemize}
  
  \item \emph{Introducci\'on al dise\~no y an\'alisis de instrumentos psicom\'etricos - Opcional}.\label{it:psic}
  
  Los instrumentos psicom\'etricos son una importante herramienta a la hora de recoger datos para investigaci\'on con personas. Se dan a conocer principios b\'asicos para la construcci\'on de pruebas, cuestionarios, etc.   
  \begin{itemize}
    \item Fundamentos de la teor\'ia cl\'asica de medici\'on.
    \item Dise\~no de instrumentos v\'alidos y confiables.
    \item Consistencia interna y an\'alisis factorial.
    \item Estandarizaci\'on. Error est\'andar de medici\'on.
  \end{itemize}

\item \emph{An\'alisis inferencial con el software R}.\label{it:r}

El software R, gratuito y libre, es una de las herramientas m\'as potentes utilizadas por estad\'isticos. El curso contempla un tutorial en que dicho software se utiliza para realizar tareas elementales de tratamiento y an\'alisis de datos.

  \begin{itemize}
    \item Introducci\'on al lenguaje R.
    \item Prueba $t$ de Student.
    \item ANOVA.
    \item Correlaci\'on.
    \item Prueba $\chi^2$
  \end{itemize}
  
  
  \item \emph{Introducci\'on a la metodolog\'ia de investigaci\'on cualitativa}.

    Desde una mirada epistemol\'ogica, los enfoques cualitativos de la investigaci\'on presentan cr\'iticas a la investigaci\'on cuantitativa. Desde un punto de vista utilitario, proporcionan herramientas \'utiles para abordar aspectos de preguntas cient\'ificas diferentes a los que abarcan los m\'etodos cuantitativos. Se presentan algunos conceptos b\'asicos y el esquema general de una investigaci\'on de corte cualitativo.
      
  \begin{itemize}
    \item Fundamentos de la investigaci\'on cualitativa.
    \item Diversidad de enfoques.
    \item El proceso de recolecci\'on de datos.
    \item An\'alisis y comunicaci\'on de resultados.
  \end{itemize}

  \item \emph{T\'opicos de cierre}
  \begin{itemize}
    \item El proceso de publicaci\'on cient\'ifica.
    \item Dilemas \'eticos asociados a la experimentaci\'on con seres humanos.
    \item Discusi\'on sobre el rol de las t\'ecnicas y m\'etodos de an\'alisis en la investigaci\'on.
  \end{itemize}
\end{enumerate}


\section{Metodolog\'ia}

Los contenidos de las secciones \ref{it:desc}, \ref{it:inf} y \ref{it:psic} se presentan mediante exposiciones te\'oricas y sesiones pr\'acticas que ense\~nan a aplicar los m\'etodos mediante el uso de herramientas computacionales. Se favorece el uso de planillas de c\'alculo por su amplia disponibilidad. En la secci\'on~\ref{it:r} se trabaja con el software R, realizando sesiones tutoriales. En las secciones restantes, los contenidos se presentan en sesiones te\'oricas, valor\'andose la discusi\'on por parte de los alumnos.

\section{Evaluaci\'on}

Durante el semestre se realizar\'an varias actividades en clase, las cuales ser\'an calificadas con nota. Las fechas est\'an indicadas en el programa, el cual aparece m\'as adelante. Aquellas actividades que requieran el uso de computador podr\'an llevarse a cabo individualmente o en pares, pero cada estudiante deber\'a entregar el resultado de su trabajo individualmente.

Estas actividades tendr\'an una ponderaci\'on equivalente a la mitad de la nota final del curso. La mitad restante corresponder\'a a la nota de una monograf\'ia que ser\'a escrita en parejas durante la segunda mitad del semestre. El tema deber\'a acordarse con el profesor en cada caso. Para este prop\'osito, los estudiantes deben elegir un t\'opico relacionado con el curso que les resulte interesante y realizar una breve investigaci\'on sobre el mismo.


\section{Bibliograf\'ia}

{\leftskip 0.5in
\parindent -0.5in

Aron, A., \& Aron, E. (2001). Estad\'istica para psicolog\'ia. Buenos Aires: Prentice Hall.

Cohen, R. J., \& Swerdlik, M. E. (2006). Pruebas y evaluaci\'on psicol\'ogicas: Introducci\'on a las pruebas y a la medici\'on. McGraw-Hill.

Hair, J. F., Anderson, R., Tathan, R. L., \& Black, W. C. (1999). An\'alisis multivariante. Madrid: Prentice Hall.

Flick, U. (2004). Introducci\'on a la investigaci\'on cualitativa. Ediciones Morata.

Verzani, J. (2005). Using R for introductory statistics. Boca Raton: Chapman \& Hall/CRC.

}

\section{Programa}

\begin{center}
\begin{tabular}{l@{\hskip 0.5in}l}
\hline
\hline \\
\chapBegin Panorama de la investigaci\'on cient\'ifica \chapEnd & \contBegin
	\item[5-Mar] Presentaci\'on. 
	\contEnd \hline \\
\chapBegin Estad\'istica descriptiva aplicada a la investigación cuantitativa. \chapEnd &  \contBegin
	\item[12-Mar] Preparaci\'on de material. Estadísticos descriptivos.
	\item [] Comunicaci\'on de resultados. Evaluaci\'on.
	\contEnd \hline \\
\chapBegin Introducción a la estadística inferencial. \chapEnd & \contBegin
	\item[19-Mar] Estrategia de an\'alisis. Prueba t de Student.
	\item[] Evaluaci\'on
	\item[26-Mar] ANOVA de un factor.
	\item[9-Abr] ANOVA de dos factores.
	\item[16-Abr] ANOVA  de medidas repetidas
	\item[] Evaluación.
	\item[23-Abr]Correlaci\'on. Prueba $\chi^2$.
	\item[] Evaluaci\'on.
	\contEnd \hline \\
\chapBegin Introducción al diseño y análisis de instrumentos de medici\'on. \chapEnd & \contBegin
	\item[7-May] Teoría clásica de medici\'on. Dise\~no. Consistencia interna.
	\item[] Análisis factorial. Estandarizaci\'on.
	\contEnd \hline \\
\chapBegin An\'alisis inferencial con el software R. \chapEnd & \contBegin
	\item[28-May] Preparaci\'on de material. Introducci\'on al lenguaje R.
	\item[] Prueba $t$ de Student. Evaluaci\'on.
	\item[4-Jun] ANOVA. Correlaci\'on. Prueba $\chi^2$
	\item[] Evaluaci\'on
	\contEnd \hline \\
\chapBegin Introducción a la metodología de investigación cualitativa. \chapEnd & \contBegin
	\item[11-Jun] Fundamentos de la investigación cualitativa. Enfoques.
	\item[] Recolecci\'on de datos. An\'alisis. Consideraciones \'eticas.
	\contEnd \hline \\
\chapBegin Otros tópicos. \chapEnd & \contBegin
	\item[18-Jun] Publicación científica. \'Etica de la investigación con seres humanos.\\
	\item[] M\'etodos y t\'ecnicas de an\'alisis en la investigaci\'on.
	\contEnd \hline \hline \\
\end{tabular}
\end{center}

\end{document}
