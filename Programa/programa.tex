\documentclass[letterpaper,11pt]{article}
\usepackage[T1]{fontenc}
\usepackage[utf8]{inputenc}
\usepackage{lmodern}
\usepackage[spanish]{babel}
\usepackage[margin=2cm]{geometry}

\title{{\Large Facultad de Letras - Pontificia Universidad Católica de Chile } \\ Curso LET3106 - Métodos y técnicas de investigación}
\author{Profesor: Esteban Hurtado}

\newcommand{\chapBegin}{\begin{minipage}[t]{0.2\columnwidth}}
\newcommand{\chapEnd}{\end{minipage}}

\newcommand{\contBegin}{\begin{minipage}[t]{0.65\columnwidth}\begin{itemize}\setlength{\itemsep}{0pt}\setlength{\parskip}{0pt}}
\newcommand{\contEnd}{\end{itemize}\end{minipage} \\[0.25in] \\}

\begin{document}

\maketitle

\section{Descripción del curso}

El curso es una introducción a la metodología de la investigación científica en lingüística. Comprende una serie de técnicas de análisis de datos de uso común a partir de las cuales se planten preguntas de investigación posibles de abordar y asuntos relativos al diseño de estudios.

El foco está puesto principalmente en técnicas de corte cuantitativo, algunas de las cuales son presentadas en detalle por su relevancia para el estudio y colaboraciones posteriores del alumno en el marco de sus propias investigaciones. Sin embargo, también se tratan técnicas cualitativas con el objetivo de reflexionar sobre la continuidad entre ambos tipos de técnica a la hora de abordar objetos de estudio propios de las ciencias humanas.

El curso no asume conocimientos técnicos o matemáticos previos. Los conceptos numéricos necesarios se desarrollan durante el curso desde la base. Se persigue dotar al alumno de un conjunto elemental de herramientas que le permita enfrentar con comodidad un amplio rango de preguntas científicas, evaluar investigaciones críticamente y enfrentar el estudio personal de tópicos nuevos en el área.

\section{Objetivos}

\begin{itemize}
  \item Entregar al alumno de herramientas elementales que le permitan enfrentar satisfactoriamente problemas de investigación y diseños metodológicos sencillos.
    
  \item Proporcionar una mirada ``desde dentro'' del proceso de diseño de estudios y análisis de datos, útil a la hora de conocer y evaluar resultados de investigaciones.

  \item Aportar con elementos que enriquezcan las opiniones que cada alumno construya sobre controversias relacionadas con la investigación científica.
  
  \item Preparar para el estudio personal de otras herramientas de análisis que se apoyan en los mismos fundamentos teóricos.
  
\end{itemize}

\section{Contenidos}

\begin{enumerate}
  \item \emph{Panorama del curso}
  
  Presentación general de los temas a abordar. Discusión acerca de la naturaleza de las preguntas lingüísticas y el rol de un nivel de análisis metodológico a la hora de buscarles respuesta. 
  
  \item \emph{Estadística descriptiva}.\label{it:desc}
  
 Introducción de conceptos importantes para el abordaje de los tópicos siguientes.
  
  \begin{itemize}
    \item Estadísticos descriptivos esenciales.
    \item Comunicación de estadísticos descriptivos.
  \end{itemize}

  \item \emph{Introducción a la estadística inferencial}.\label{it:inf}
  
La estadística inferencial es el conjunto de herramientas predilecto a la hora de juzgar si los patrones formales que se observan en una muestra son representativos de lo que ocurre en la población general. Se presentan las técnicas básicas y se muestra cómo ayudan a responder preguntas.

  \begin{itemize}
    \item Estrategia de análisis en estadística inferencial.
    \item Contraste de dos muestras con la prueba $t$ de Student.
    \item Análisis de datos en diseños experimentales más sofisticados mediante análisis de la varianza.
    \item Prueba $\chi^2$
    \item Visualización de datos.
    \item Comunicación de resultados
  \end{itemize}
  
  \item \emph{Introducción al diseño y análisis de instrumentos psicométricos - Opcional}.\label{it:psic}
  
  Los instrumentos psicométricos son una alternativa útil para recoger datos en la investigación de seres humanos. Se presentan conceptos y principios básicos para la construcción de pruebas, cuestionarios, etc.   

  \begin{itemize}
    \item Fundamentos de la teoría clásica de medición.
    \item Diseño de instrumentos válidos y confiables.
    \item Consistencia interna y análisis factorial.
    \item Estandarización. Error estándar de medición.
  \end{itemize}

\item \emph{Análisis inferencial con el software R}.\label{it:r}

Un paquete de software estadístico es un ítem importante en la caja de herramientas de un analista de datos. En particular, el software R es muy poderoso y goza de la mejor reputación. Es gratuito y de código abierto. Esto le permite satisfacer criterios de neutralidad y transparencia, y facilitar la colaboración, lo cual lo hace óptimo para su uso en la investigación científica. Todo lo anterior conlleva el precio de una curva de aprendizaje algo pronunciada, razón por la que el curso dedica espacio a tratar sus rudimentos.

  \begin{itemize}
    \item Introducción al lenguaje R.
    \item Exploración de conjuntos de datos.
    \item Prueba $t$ de Student.
    \item ANOVA.
    \item Correlación.
    \item Prueba $\chi^2$
    \item Visualización de datos y reporte.
  \end{itemize}
  
  
  \item \emph{Introducción a la metodología de investigación cualitativa}.

    Desde una mirada metodológica, hay enfoques cualitativos de la investigación que parecen presentar críticas a la investigación cuantitativa. Sin embargo, tomando completa distancia de las posibles críticas, la mirada cualitativa es una opción importante, a veces esencial para buscar solución a ciertas preguntas científicas. En términos prácticos, las técnicas cualitativas pueden llevarse a cabo en diferentes momentos de un estudio ya sea de manera principal o para apoyar el avance de un proyecto que involucre otras técnicas. Se trata de protocolos que sistematizan la desafiante labor de usar la interpretación como instrumento para hacer ciencia.
      
  \begin{itemize}
    \item Fundamentos de la investigación cualitativa.
    \item Diversidad de enfoques.
    \item El proceso de recolección de datos.
    \item Análisis y comunicación de resultados.
  \end{itemize}

  \item \emph{Tópicos de cierre}
  \begin{itemize}
    \item El proceso de publicación científica.
    \item Dilemas éticos en relación a los seres humanos que participan en un estudio.
    \item Propiedad de los datos, los materiales y los reportes.
    \item Espíritu y buena aplicación de principios éticos en investigación.
    \item La responsabilidad del investigador.
  \end{itemize}
\end{enumerate}


\section{Metodolog\'ia}

Los contenidos de las secciones \ref{it:desc}, \ref{it:inf} y \ref{it:psic} se presentan mediante exposiciones te\'oricas y sesiones pr\'acticas que ense\~nan a aplicar los m\'etodos mediante el uso de herramientas computacionales. Se favorece el uso de planillas de c\'alculo por su amplia disponibilidad. En la secci\'on~\ref{it:r} se trabaja con el software R, realizando sesiones tutoriales. En las secciones restantes, los contenidos se presentan en sesiones te\'oricas, valor\'andose la discusi\'on por parte de los alumnos.

\section{Evaluaci\'on}

Durante el semestre se realizan varias actividades en clase, cada una de las cuales se califica con una nota. Las fechas están indicadas en el calendario que aparece m\'as adelante. Aquellas actividades que requieren el uso de computador se pueden realizar individualmente o en pares, entregando cada estudiante el resultado de su trabajo individualmente.

Estas actividades tienen una ponderación equivalente a la mitad de la nota final del curso. La mitad restante corresponde a la nota de una monograf\'ia que se escribe en parejas durante la segunda mitad del semestre. El tema debe acordarse con el profesor en cada caso. Para este prop\'osito, los estudiantes deben elegir un t\'opico relacionado con el curso que les resulte interesante y realizar una breve investigaci\'on sobre el mismo.


\section{Bibliograf\'ia}

{\leftskip 0.5in
\parindent -0.5in

Aron, A., \& Aron, E. (2001). Estad\'istica para psicolog\'ia. Buenos Aires: Prentice Hall.

Cohen, R. J., \& Swerdlik, M. E. (2006). Pruebas y evaluaci\'on psicol\'ogicas: Introducci\'on a las pruebas y a la medici\'on. McGraw-Hill.

Hair, J. F., Anderson, R., Tathan, R. L., \& Black, W. C. (1999). An\'alisis multivariante. Madrid: Prentice Hall.

Flick, U. (2004). Introducci\'on a la investigaci\'on cualitativa. Ediciones Morata.

Verzani, J. (2005). Using R for introductory statistics. Boca Raton: Chapman \& Hall/CRC.

}


\end{document}
