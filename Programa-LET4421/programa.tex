\documentclass[letterpaper,11pt]{article}
\usepackage[T1]{fontenc}
\usepackage[utf8]{inputenc}
\usepackage{lmodern}
\usepackage[spanish]{babel}
\usepackage[margin=2cm]{geometry}

\title{{\Large Facultad de Letras - Pontificia Universidad Católica de Chile } \\ Curso LET4421-1 - Investigación en ciencias del lenguaje}
\author{Profesor: Esteban Hurtado}

\newcommand{\chapBegin}{\begin{minipage}[t]{0.2\columnwidth}}
\newcommand{\chapEnd}{\end{minipage}}

\newcommand{\contBegin}{\begin{minipage}[t]{0.65\columnwidth}\begin{itemize}\setlength{\itemsep}{0pt}\setlength{\parskip}{0pt}}
\newcommand{\contEnd}{\end{itemize}\end{minipage} \\[0.25in] \\}

\begin{document}

\maketitle
%\tableofcontents

%\begin{abstract}
%\end{abstract}

\section{Descripci\'on del curso}

Ante la tarea de responder preguntas científicas y satisfacer los objetivos de un proyecto de investigación, existe un conjunto amplio de herramientas capaces de aportar claridad y de reducir la complejidad procedimental y conceptual. En este contexto, las técnicas cuantitativas de análisis de datos son relevantes pues señalan aquellas relaciones formales que se pueden estudiar en las observaciones de un fenómeno, de manera relativamente directa y con un nivel de esfuerzo menor que en el caso de otros tipos de técnicas. Es por esto que forman parte esencial de la ``caja de herramientas'' del investigador, ya sea para usarse como principal instrumento de generación de conocimiento o para plantearse como un apoyo en proyectos de corte hermenéutico.

Este curso aborda contenidos actualizados sobre herramientas cuantitativas para la investigación en ciencias humanas y especialmente en ciencias del lenguaje. Esto sirve como núcleo desde el cual problematizar los diseños, procedimientos y materiales que dan vida a una investigación científica. Conocimientos previos sobre estadística descriptiva e inferencial son deseables pero no imprescindibles.

\section{Objetivos}

\begin{itemize}
  \item Dotar al alumno de las herramientas cuantitativas más relevantes para apoyar la investigación en ciencias del lenguaje.
    
  \item Capacitar al alumno para seleccionar y llevar a cabo análisis estadísticos comunes de manera correcta y en consistencia con sus preguntas de investigación.
  
  \item Favorecer en el alumno el análisis crítico de procedimientos investigativos, con el fin de producir diseños coherentes con las herramientas de análisis disponibles.
  
  \item Proporcionar los conocimientos necesarios para comprender la metodología de investigaciones cuantitativas actuales en ciencias del lenguaje.


\end{itemize}

\section{Descripción de los contenidos}

\begin{enumerate}
  \item \emph{Introducción}
  
  Se discute la naturaleza de las técnicas comprendidas en el curso, situándolas en una mirada científica de fenómenos del lenguaje.
    
  \item \emph{Estad\'istica descriptiva}.\label{it:desc}
  
  Se revisan contenidos básicos para beneficio de estudiantes que no posean conocimientos previos, introduciendo a la vez conceptos que serán ampliamente utilizados posteriormente en el curso.

  \item \emph{Estad\'istica inferencial}.\label{it:inf}
  
  En una modalidad de tutorial, esta sección aborda una serie de herramientas estadísticas fundamentales en la investigación actual. Cada exposición teórica va seguida de una actividad práctica en clase. Esto tiene el fin de que, con la guía del profesor, el alumno tenga una experiencia en primera persona con los procedimientos estadísticos y pueda comprender su aplicabilidad. Se trata de herramientas que facilitan la detección de patrones complejos y sutiles en los datos de investigaciones, y que proporcionan criterios para determinar si es razonable generalizar lo observado en una muestra a la población que la originó.
  \item \emph{Análisis multivariado de datos}.
  
  Se presentan métodos que facilitan la confirmación o el descubrimiento de dimensiones de variación presentes en un conjunto de observaciones. Ello permite agrupar elementos, conceptos o personas y estudiar relaciones.
  
  \item \emph{Análisis de series de tiempo}.
  
  Se introduce el análisis de magnitudes que cambian en el tiempo. Las aplicaciones van desde investigaciones que analizan registros de datos que se extienden por años, hasta el estudio de procesos que duran segundos o menos (e.g. fenómenos fonéticos).
  
  \item \emph{Técnicas computacionales}.

 Si bien en la actualidad casi todos los proyectos de investigación involucran el uso de computadores, hay ciertas actividades que que están ligadas a dicha tecnología de manera esencial. Se hace una introducción a algunas técnicas importantes.

  \item \emph{Tópicos de cierre}.

 Esta sección tiene el fin de dar una mirada a las motivaciones iniciales del curso, a la luz de los nuevos conocimientos adquiridos. Se enfatiza la necesidad de mantener siempre a la vista las preguntas que guían las investigaciones. Se presentan los métodos cuantitativos como un medio para realizar la labor mucho más amplia de generar conocimientos. Se reflexiona sobre los desafíos éticos que surgen a partir de investigaciones que se apoyan fuertemente en el análisis de datos provenientes de seres humanos.
 
\end{enumerate}

\section{Metodolog\'ia}

Las clases consisten en exposiciones teóricas seguidas de actividades prácticas. Cuando sea pertinente se entregará material en texto o video para el estudio personal de algunos de los métodos estudiados en las clases.

\section{Evaluación}

El promedio de las calificaciones de las actividades en clase tendrá una ponderación de dos tercios de la nota final. El tercio adicional corresponderá a la calificación de un trabajo que podrá realizarse individualmente o en parejas. Este consistirá en escribir detalladamente el diseño metodológico de una investigación. Los detalles deberán acordarse con el profesor.

\section{Bibliografía}

{\leftskip 0.5in
\parindent -0.5in

Aron, A., \& Aron, E. (2001). Estadística para psicología. Buenos Aires: Prentice Hall.

Hair, J. F., Anderson, R., Tathan, R. L., \& Black, W. C. (1999). Análisis multivariante. Madrid: Prentice Hall.

Smith, S. W. (1997). The scientist and engineer's guide to digital signal processing.

Venables, W. N., Smith, D. M., \& R Development Core Team. (2002). An introduction to R.

}



\end{document}
